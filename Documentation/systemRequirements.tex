\chapter{System Dependencies}
  \subsection{For Linux:}
  \subsubsection*{To get QT:}
    \begin{itemize}
      \item
    \end{itemize}
  \subsubsection*{To get C++ \& GCC:}
    \begin{itemize}
      \item
    \end{itemize}
  \subsubsection*{To get Python3:}
    \begin{itemize}
      \item
    \end{itemize}

  \bigskip
  \subsection{For Windows:}
  \subsubsection*{To get QT:}
    \begin{itemize}
      \item
    \end{itemize}
  \subsubsection*{To get C++ \& GCC:}
    \begin{itemize}
      \item
    \end{itemize}
  \subsubsection*{To get Python3:}
    \begin{itemize}
      \item
    \end{itemize}

  \bigskip
  \subsection{For MacOS:}
  \subsubsection*{To get QT:}
    \begin{itemize}
      \item Go to https://www.qt.io click on "Download. Try."
      \item Scroll down and click on "Go open source"
      \item Scroll down and click on "Download the Qt Online Installer"
      \item Click "Download"
    \end{itemize}
  \subsubsection*{To\ get C++ \& GCC:}
    \begin{itemize}
      \item Open the terminal app
      \item Type g++ $<$filename$>$
      \item if prompted download mac's develper tools
    \end{itemize}
  \subsubsection*{To get Python3:}
    \begin{itemize}
      \item Open terminal type: /bin/bash -c \begin{verbatim} "$(curl -fsSL https://raw.githubusercontent.com/Homebrew/install/master/install.sh)" \end{verbatim}
      \item At the bottom of your "~/.profile" file  type: \begin{verbatim} export PATH="/usr/local/opt/python/libexec/bin:$PATH" \end{verbatim}
      \item type "brew install python"
    \end{itemize}
