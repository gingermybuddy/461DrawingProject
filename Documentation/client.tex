\chapter{Client}
  The following is an explanation of the client side of the 461 Drawing Project \linebreak \linebreak

\section{ItemStats}
\subsection{.h}
  itemStats is a struct which contains all of the values which a shape/object
contains. This is used in other documents and functions to help parse out and
separate data.
\begin{minted}
[
frame=lines,
framesep=2mm,
baselinestretch=1.2,
fontsize=\footnotesize,
]
{cpp}
struct itemStats
{
	//Lines get tracked the same as everything else.
	//Set the width/height variables to x2 and y2, respectively.
	std::string board_id;
        std::string type;
        int id;
        double x;
        double y;
        double height;
        double width;
        QColor rgb;

	//Functions to turn the thing into a QJsonObject
	//and a QByteArray that can be sent directly over a socket.

  QJsonObject toJson();
	QByteArray byteData();

	//Several different constructors for a couple of different
	//uses. You can pass in nothing, a QGraphicsItem* for the
	//client end, or a bunch of variables that it will simply
	//insert.
	itemStats();
	itemStats(std::string board_id, QGraphicsItem* item);
	itemStats(std::string board_id, std::string type, int id, double x, double y, \
  double height, double width, QColor rgb);
  ~itemStats();
};
\end{minted}

\subsection{.cpp}
\begin{minted}
[
frame=lines,
framesep=2mm,
baselinestretch=1.2,
fontsize=\footnotesize,
]
{cpp}
itemStats::itemStats(std::string nboard_id, std::string ntype, int nid, double nx, double ny, \
double nheight, double nwidth, QColor nrgb)
  //UPDATE THIS IF YOU ADD THINGS TO THE PLACED ITEMS

itemStats::itemStats(std::string nboard_id, QGraphicsItem* item)
  /*UPDATE THIS IF YOU ADD THINGS TO THE PLACED ITEMS
  Parses the QGraphicsItem and throws the variables into their appropriate places.*/

itemStats::~itemStats()
  //destructor for itemStats

QJsonObject itemStats::toJson()
  /*UPDATE THIS IF YOU ADD THINGS TO THE PLACED ITEMS
  Turns the item into a JSON object, formatted as it would be
  in the jsonExamples folder. This doesn't actually differ at
  all from different shape types.

  For the 'end' parameter, it uses the width and height as x and y coordinates.
  If the thing's a line, the end is just what it says on the tin - the ending coordinates of the line. */

QByteArray itemStats::byteData()
  /*Turns it first into a JSON object, then passes that object
	into a QByteArray. This array can be sent straight to the server.*/

\end{minted}

\section{ProjectScene}
\subsection{.h}
  projectScene is a class which contains functions that relate to the QGraphicsScene,
that is, the 'behinds the scenes' of the graphical window.
\begin{minted}
[
frame=lines,
framesep=2mm,
baselinestretch=1.2,
fontsize=\footnotesize,
]
{cpp}
class ProjectScene : public QGraphicsScene
{
	Q_OBJECT
	private:
		QTcpSocket* m_socket;
		std::vector<itemStats> m_tracked_items;
		std::string m_board_id;

	public slots:
		void sceneChanged(const QList<QRectF> &region);
		void readSocket();
		void disconnect();
	signals:
	public:
                int trackItem(QGraphicsItem* item);
		ProjectScene();
		~ProjectScene();
};

\end{minted}

\subsection{.cpp}
\begin{minted}
[
frame=lines,
framesep=2mm,
baselinestretch=1.2,
fontsize=\footnotesize,
]
{cpp}
  ProjectScene::ProjectScene()
    //A constructor for the ProjectScene class

  ProjectScene::~ProjectScene()
    //A destructor for the ProjectScene class

  ProjectScene::readSocket()
    //Reads in data that is received from the server into a JSON object

  ProjectScene::disconnect()
    /*Deletes the socket from the scene, effectively removing the connection from
      the server.*/

  ProjectScene::trackItem()
    /*Pushes back all the data from a received graphics item into an internal list
      of graphics items.*/

  ProjectScene::sceneChanged()
    /*When something in the scene changes, this event is triggered. What we do is
     get a list of items that have changed (compare between the last update and
     our internal list) and then tells the server about it.*/
\end{minted}

\section{ProjectView}
\subsection{.h}
  projectView is a class which contains the functions which relate to the QGraphicsView,
that is, the canvas and buttons that we see on the graphical window.
\begin{minted}
[
frame=lines,
framesep=2mm,
baselinestretch=1.2,
fontsize=\footnotesize,
]
{cpp}
class ProjectView : public QGraphicsView
{
	Q_OBJECT
	private:
		int m_tool;
		int m_color_r;
		int m_color_g;
		int m_color_b;

        QPointF firstClick;
	private slots:

	public:
		void change_tool(int tool);
		void change_color(int r, int g, int b);

        void circle_tool(qreal x, qreal y,qreal x2, qreal y2);
        void line_tool(qreal x, qreal y, qreal x2, qreal y2);
		void rect_tool(qreal x, qreal y, qreal x2, qreal y2);
		void mousePressEvent(QMouseEvent* event);
		void mouseReleaseEvent(QMouseEvent* event);
		ProjectView();
		~ProjectView();
};
\end{minted}


\subsection{.cpp}

\begin{minted}
[
frame=lines,
framesep=2mm,
baselinestretch=1.2,
fontsize=\footnotesize,
]
{cpp}
ProjectView::ProjectView()
  //A constructor for the ProjectView class

ProjectView::~ProjectView()
  //A destructor for the ProjectView class

ProjectView::change_tool(int tool)
  //Change the current tool based on an integer parameter

ProjectView::change_color(int r, int g, int b)
    //Change the current tool's color based on 3 integer parameters, r, g, and b.

ProjectView::circle_tool(qreal x, qreal y, qreal x2, qreal y2)
  /*Draw a circle on the canvas using two points, (x,y) and (x2,y2). This allows
    us to draw ellipses easily as well as circeles, similarly to how we draw
    rectangles.
    (x,y) is the first location the user clicked, and (x2,y2) is the second.*/

ProjectView::line_tool(qreal x, qreal y, qreal x2, qreal y2)
    /*Draw a line on the canvas using two points, (x,y) and (x2,y2).
    (x,y) is the first location the user clicked, and (x2,y2) is the second.*/

ProjectView::rect_tool(qreal x, qreal y, qreal x2, qreal y2)
    /*Draw a rectangle on the canvas using two points, (x,y) and (x2,y2).
    (x,y) is the first location the user clicked, and (x2,y2) is the second.*/

ProjectView::mousePressEvent()
     /*This event is triggered when the user clicks down on the mouse. Technically,
     it stores this point and doesn't do anything with it until a mouseReleaseEvent.*/

ProjectView::mouseReleaseEvent()
    /*This event is triggered when the user releases the click on the mouse. This
    function takes the second point, figures out which action to take based on
    the selected tool, and calls the appropriate function.*/
\end{minted}

\section{ToolBar}
\subsection{.h}
  toolbar is a class which contains the functions to change the different tools in
the graphical window.
\begin{minted}
[
frame=lines,
framesep=2mm,
baselinestretch=1.2,
fontsize=\footnotesize,
]
{cpp}
class ToolBar : public QWidget
{
	Q_OBJECT
	private:
		QPushButton* m_circle;
		QPushButton* m_line;
		QPushButton* m_rect;
		QPushButton* m_default;
		QPushButton* m_black;
		QPushButton* m_red;
		QPushButton* m_green;
		QPushButton* m_yellow;
		QPushButton* m_blue;
		QPushButton* m_color_picker;
		QVBoxLayout* m_layout;
		ProjectView* m_view;
	public slots:
		// set default
        	void set_default();
        	// shapes
        	void place_rectangle();
        	void set_line();
        	void set_circle();
        	// colors
        	void set_color_black();
        	void set_color_red();
        	void set_color_green();
        	void set_color_yellow();
        	void set_color_blue();
        	void set_color_custom();

	public:
		ToolBar();
		~ToolBar();
		void set_view(ProjectView* view);

};
\end{minted}

\subsection{.cpp}
\begin{minted}
[
frame=lines,
framesep=2mm,
baselinestretch=1.2,
fontsize=\footnotesize,
]
{cpp}
ToolBar::ToolBar()
  //A constructor for the ToolBar class

ToolBar::~ToolBar()
  //A destructor for the ToolBar class

ToolBar::set_view(ProjectView* view)
  //Sets the tool's m_view to view

ToolBar::set_default()
  //Sets the default tool to the select tool

ToolBar::set_line()
  //Sets the tool to the line tool

ToolBar::set_circle()
  //Sets the tool to the circle tool

ToolBar::set_rect()
  //Sets the tool to the rect tool

ToolBar::set_color_black()
  //Sets the tool's color to be black

ToolBar::set_color_red()
  //Sets the tool's color to be red

ToolBar::set_color_green()
  //Sets the tool's color to be green

ToolBar::set_color_yellow()
  //Sets the tool's color to be yellow

ToolBar::set_color_blue()
  //Sets the tool's color to be blue



ToolBar::set_color_custom()
  /*Brings up a color picker menu for the user to set their own color using a
    graphical interface, rgb, hexadeicimal, or cmyk.*/
\end{minted}

\section{Window}
\subsection{.h}
  window is essentially the 'main' of the graphical window, though there is a
'main' that spawns it.
\begin{minted}
[
frame=lines,
framesep=2mm,
baselinestretch=1.2,
fontsize=\footnotesize,
]
{cpp}
class Window : public QMainWindow //Extension on the base QMainWindow class
{
	Q_OBJECT //Must be included for qmake to recognize this

	private: //Menus and features of the window
		ToolBar* m_bar;
		QDockWidget* m_tool_dock;
		ProjectScene* m_scene;
		ProjectView* m_view;

        // shape tools

	private slots: //Where functions attached to buttons go
        // make a popup window
		void popup();
	public:
		Window();
		~Window();
};
\end{minted}

\subsection{.cpp}
\begin{minted}
[
frame=lines,
framesep=2mm,
baselinestretch=1.2,
fontsize=\footnotesize,
]
{cpp}
Window::Window()
  /*A constructor for the Window class. Sets up the entire graphical window with
    toolbars, docks, a canvas, a scene, and a menu.*/

Window::~Window()
  //A destructor for the Window class.
\end{minted}

\section{Main}
\subsection{.cpp}
  main.cpp is a small file which contains one function: int main. This is the start of
the entire client, where a new window is created in memory and executed. From
there, upon returning, the event loop begins.
\begin{minted}
[
frame=lines,
framesep=2mm,
baselinestretch=1.2,
fontsize=\footnotesize,
]
{cpp}
int main(int argc, char* argv[])
  /*This is the start of the entire client, where a new window is created in memory and executed.*/
\end{minted}
